\documentclass[a4paper,10pt]{article}

\usepackage{amsmath}
\usepackage{mathtools}
\usepackage{comment}
\usepackage[pdfborder={0 0 0}]{hyperref}
\usepackage[utf8]{inputenc}
\usepackage{listings}

\lstset{
    language=C++,
    basicstyle=\ttfamily,
    captionpos=b,
    breaklines=true,
    breakatwhitespace=false,
    showspaces=false,
    showtabs=false,
    numbers=left,
}

\title{Aufgabe 3: Konstantenpropagation mit Bedingungen \\
       Optimierende Übersetzer WS 2014/2015 \\
       Technical University of Vienna}
\author{Jakob Gruber, 0203440 \\
        Mino Sharkhawy, 1025887}

\begin{document}

\maketitle

\section{Behandlung von Schleifen}

Nachdem Konstantenpropagation kein Loop Unrolling durchf\"uhrt, kann der Returnwert
folgendes Programms nicht berechnet werden:

\begin{lstlisting}
int main() {
    int i = 0;
    while (i < 1) {
        i = i + 1;
    }
    return i;
}
\end{lstlisting}

Falls allerdings der K\"orper der \lstinline|while| Schleife unerreichbar w\"are,
k\"onnte Konstantenpropagation den endg\"ultigen Wert von \lstinline|i| berechnen.
Das w\"are zum Beispiel der Fall, wenn initial \lstinline|i = 1| gesetzt wird.

\section{Konstantenpropagation mit Bedingungen mit PAG}

Das folgende PAG-Programm implementiert Konstantenpropagation mit Bedingungen:

\lstinputlisting{../constprop/constprop.optla}

\end{document}
