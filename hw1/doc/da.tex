\documentclass[a4paper,10pt]{article}

\usepackage{amsmath}
\usepackage{mathtools}
\usepackage{comment}
\usepackage[pdfborder={0 0 0}]{hyperref}
\usepackage[utf8]{inputenc}
\usepackage{listings}

\lstset{
    language=C++,
    basicstyle=\ttfamily,
    captionpos=b,
    breaklines=true,
    breakatwhitespace=false,
    showspaces=false,
    showtabs=false,
    numbers=left,
}

\title{Aufgabe 1: Dominator-Analyse \\
       Optimierende Übersetzer WS 2014/2015 \\
       Technical University of Vienna}
\author{Jakob Gruber, 0203440 \\
        Mino Sharkhawy, 1025887}

\begin{document}

\maketitle

\section{Intraprozedurale Dominator-Analyse}

In der Dominator-Analyse ist das Killset für alle Blöcke leer und das Genset enthält
nur das Blocklabel selbst. $DA_{\circ}$ wird aus dem Durchschnitt der $DA_{\bullet}$
aller Vorgänger gebildet.

\begin{align*}
        gen_{DA}(B^\ell) &= \{\ell\} & B^\ell \in blocks(S_\star) \\
        kill_{DA}(B^\ell) &= \{\} & B^\ell \in blocks(S_\star) \\
        DA_{\circ}(\ell) &=
        \begin{dcases}
                \{\} & \text{if } \ell = init(S_\star) \\
                \cap \{DA_\bullet(\ell') | (\ell', \ell) \in flow(S_\star)\} &
                \text{otherwise}\\
        \end{dcases}\\
        DA_{\bullet}(\ell) &= (DA_{\circ}(\ell) \setminus kill_{DA}(B^\ell)) \cup
        gen_{DA}(B^\ell) & B^\ell \in blocks(S_\star)
\end{align*}

\section{Intraprozedurale Dominator-Analyse mit PAG}

Die folgende PAG-Lösung enthält keine Fehlerbehandlung.

\lstinputlisting{dominators.optla}

\end{document}
