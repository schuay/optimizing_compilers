\documentclass[a4paper,10pt]{article}

\usepackage{amsmath}
\usepackage{mathtools}
\usepackage{comment}
\usepackage[pdfborder={0 0 0}]{hyperref}
\usepackage[utf8]{inputenc}
\usepackage{listings}

\lstset{
    language=C++,
    basicstyle=\ttfamily,
    captionpos=b,
    breaklines=true,
    breakatwhitespace=false,
    showspaces=false,
    showtabs=false,
    numbers=left,
}

\title{Aufgabe 2: Einfache Konstantenpropagation \\
       Optimierende Übersetzer WS 2014/2015 \\
       Technical University of Vienna}
\author{Jakob Gruber, 0203440 \\
        Mino Sharkhawy, 1025887}

\begin{document}

\maketitle

\section{Einfache intraprozedurale Konstantenpropagation}

Hier sind einige Ausdrücke aufgelistet, welche, unabhängig von der unbekannten
Variable $a$, einen konstanten Wert haben. In der linken Spalte sind arithmetische
und in der Mitte boolesche Ausdrücke. Rechts sind Bitoperationen, wobei davon
ausgegangen wird, dass die Variable $a$ vorzeichenlos und 16 Bit lang ist.

\begin{align*}
        a * 0 &= 0 & a \wedge false &= false & a \mathbin{\char`\&} 0 &= 0\\
        a - a &= 0 & a \vee true &= true & a \mathbin{\char`\|} 0xFFFF &= 0xFFFF\\
        \frac{a}{a} &= 1 & a \vee \neg a &= true & a \mathbin{\char`\^} a &= 0\\
\end{align*}

In dem folgenden C-Programm kann der Wert von $c$ nicht mit einfacher
Konstantenpropagation bestimmt werden. Würde man $c$ jeweils am Ende eines Zweigs der
If-Anweisung bestimmen, erhält man in allen Fällen $c = 5$. Nach dem If gibt es aber
zwei unterschiedliche mögliche Werte für $a$ und $b$, welche jetzt nicht mehr als
Konstanten angenommen werden können. Nachdem keiner der Operanden für $c = a + b$
bekannt ist, kann für $c$ kein konstanter Wert bestimmt werden.

\begin{lstlisting}
int main(int argc, int **argv)
{
        int a,b,c;

        if (argc > 2) {
                a = 2;
                b = 3;
        } else {
                a = 3;
                b = 2;
        }

        c = a + b;

        return c;
}
\end{lstlisting}

\section{Einfache intraprozedurale Konstantenpropagation mit PAG}

Das folgende PAG-Programm implementiert einfache Konstantenpropagation für $SL_1$.

\lstinputlisting{../constprop/constprop.optla}

\end{document}
